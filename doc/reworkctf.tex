\documentclass{article}
\usepackage[bookmarks,colorlinks=true,linkcolor=blue,urlcolor=blue]{hyperref}
\usepackage{url}
\usepackage{tabularx}
\usepackage{graphicx}
\usepackage{placeins}

\setcounter{tocdepth}{2}

\begin{document}

\title{\vspace*{\fill}PCB Rework CTF}
\author{Andrew Zonenberg\\
azonenberg@drawersteak.com}
\date{\today}

\maketitle
\begin{abstract}
The ``Rework CTF" is a game/training aid, similar to an infosec challenge-based CTF, but focusing on PCB rework and
soldering skills. The intended focus is on technician skills (fixing defects) not engineering skills (finding defects).
The game is intended to be played on fully populated boards; the provided FPGA bitstream will automatically grade most
problems and light up an LED upon successful completion. Additional blank boards may be used for practice and
instruction before a for-point challenge is attempted on a populated PCB.
\end{abstract}
\thispagestyle{empty}

\pagebreak

\tableofcontents

\pagebreak
\section{Introduction}

The bugs are sorted by difficulty, ranging from 100 to 500 points. Any and all tools are in scope, however cheating by
simply shorting the LEVELx\_OK signal to power is not allowed. Specific rules, if applicable, are described in the
subsections for each bug.

Signals related to each bug are prefixed with ``LEVELx\_" in the KiCAD design files to make the bugs easier to find.

There are two kinds of bugs in this board. Electrical connectivity bugs have numeric level identifiers (1, 2, 3, etc).
The corresponding LED will light up when these problems are successfully completed. Mechanical/visual bugs have
alphabetical level identifiers (A, B, C, etc). These are graded by looking at the board and deciding if it looks OK.

All design files, including the Verilog source for the grader bitstream, are available under a 3-clause BSD-style
license.

\pagebreak
\section{100-Point Bugs}

\subsection{Level 1}

Input to a SOIC-packaged 7404 is shorted to the adjacent Vdd. Clear the short.

\subsection{Level 2}

LED-drive enable is always high, but the LED itself was installed backwards, Use hot air to flip it around.

\subsection{Level 3}

Input pad on a SOIC-packaged 7404 footprint is missing. Restore the open connection and install the IC. (50 bonus
points for style if you put in a new SMT pad rather than just running a jumper.)

\subsection{Level 4}

0805 resistor is missing soldermask apertures. Remove the mask and solder a low-valued (anything under 1K should be fine)
resistor onto the pads.

\subsection{Level A}

The resistor you installed in level 4 doesn't have a refdes. Label it ``R30" with white ink.

\subsection{Level B}

A region of the board is missing some solder mask. Mix up some 2-part epoxy, tint it to match the existing mask, and
brush it on.

\pagebreak
\section{200-Point Bugs}

\subsection{Level 5}

Tiny short circuit between two closely spaced signals. Clear the short without damaging either line.

\subsection{Level 6}

Large short circuit between two closely spaced signals. Clear the short without damaging either line.

\subsection{Level 7}

LVDS pair on outer layer with P/N swapped. Hook them up correctly.

\subsection{Level 8}

Several pads on a TSSOP-packaged 7404 footprint are missing. Restore the open connection and install the IC. (50
bonus points for style if you put in a new SMT pad rather than just running a jumper.)

\subsection{Level 9}

Install an 01005 resistor (anything under 1K should be fine) onto the provided pads.

\subsection{Level 10}

Via connecting to input of TSSOP-packaged 7404 doesn't actually connect to the input. Fix it.

\subsection{Level 11}

Missing thermal relief connecting to one side of an 0201 passive causing tombstoning. Cut out a thermal relief and fix
the tombstoned part.

\pagebreak
\section{300-Point Bugs}

\subsection{Level 12}

A through-board via wasn't drilled. Fix it.

\subsection{Level 13}

Two signals on an internal layer are shorted. Clear the short.

\subsection{Level 14}

A dogbone via on the FPGA footprint wasn't connected to the corresponding ball. Fix it before installing the FPGA.

\pagebreak
\section{400-Point Bugs}

\subsection{Level 15}

A differential pair on an internal layer is swapped. Fix it. For this level, you're not allowed to just drill out the
trace and connect the vias directly, we're pretending the vias are blind/buried. You must fix this with an inner-layer
circuit edit.

\subsection{Level 16}

Two signals on an internal layer are shorted together. Clear the short. Heavy routing congestion on the top layer
make this harder than level 13.

\subsection{Level 17}

Due to a drill-gerber file version mismatch, a signal via was shorted to the internal ground plane. Clear the short.

\subsection{Level 18}

A dogbone via on the FPGA footprint wasn't drilled out. Fix it before installing the FPGA.

\pagebreak
\section{500-Point Bugs}

\subsection{Level 19}

Two blind vias at the same place on opposite sides of the board were incorrectly drilled as a through-board via.
Separate the two vias so that the outer layers connect to the adjacent plane, but they don't short to each other.

\subsection{Level 20}

A strap pin under a BGA was incorrectly tied high. Pull it low instead.

\subsection{Level 21}

Two signals on an internal layer are shorted together. Clear the short. Heavy routing congestion on both outer layers
make this harder than level 16.

\subsection{Level 22}

A dogbone via on the FPGA footprint wasn't drilled out. Fix it after installing the FPGA.

\subsection{Level 23}

I put 22 intentional bugs in this board. Can you find a 23rd? If so, there's a spot marked for you to install another
LED. You've earned it!

\end{document}

